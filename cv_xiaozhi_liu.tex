%%%%%%%%%%%%%%%%%%%%%%%%%%%%%%%%%%%%%%%%%
% Xiaozhi Liu - Professional CV
% Modern & Clean Design for Tech Industry
%%%%%%%%%%%%%%%%%%%%%%%%%%%%%%%%%%%%%%%%%

\documentclass[11pt,a4paper,sans]{moderncv}

% ModernCV themes
\moderncvstyle{banking} % Style: 'casual', 'classic', 'banking', 'oldstyle', 'fancy'
\moderncvcolor{blue}    % Color: 'blue', 'orange', 'green', 'red', 'purple', 'grey', 'black'

% Character encoding
\usepackage[utf8]{inputenc}

% Adjust page margins
\usepackage[scale=0.85]{geometry}
\setlength{\hintscolumnwidth}{3cm} % Width of the timeline on the left

% Personal data
\name{Xiaozhi}{Liu}
\phone[mobile]{(+86)~135~9300~4230}
\email{xzliu@buaa.edu.cn}
\homepage{xzliu-opt.github.io}
\social[github]{xzliu-opt}

% Optional: uncomment if you want to add these
% \social[linkedin]{xiaozhi-liu}
% \extrainfo{Additional information}

%%%%%%%%%%%%%%%%%%%%%%%%%%%%%%%%%%%%%%%%%
\begin{document}

\makecvtitle

%%%%%%%%%%%%%%%%%%%%%%%%%%%%%%%%%%%%%%%%%
\section{Education}

\cventry{Sept. 2022 -- Present}{Ph.D. in Applied Mathematics}{Beihang University}{Beijing, China}{}{
  \begin{itemize}
    \item School of Mathematical Sciences \& \textbf{Shen Yuan Honors College} (selected among only \textbf{35 students university-wide})
    \item Supervisor: Prof. Yong Xia
    \item GPA: \textbf{91.45/100} (Rank: \textbf{4/27})
  \end{itemize}
}

\cventry{Sept. 2018 -- Jul. 2022}{B.S. in Information and Computing Science}{Northwestern Polytechnical University}{Xi'an, China}{}{
  \begin{itemize}
    \item School of Mathematics and Statistics
    \item Supervisor: Prof. Jianchao Bai
    \item GPA: \textbf{88.01/100} (Rank: \textbf{4/43})
    \item During my junior year, I achieved a GPA of \textbf{97.89/100}, ranking \textbf{first in the entire college (1/104)}
  \end{itemize}
}

\cventry{Oct. 2025 -- Oct. 2026}{Visiting Ph.D. in Applied Mathematics}{Université catholique de Louvain}{Louvain-la-Neuve, Belgium}{}{
  \begin{itemize}
    \item Institute of Information and Communication Technologies, Electronics and Applied Mathematics (INMA/ICTEAM)
    \item Supervisor: Prof. Geovani N. Grapiglia
    \item Funding: Supported by the \textbf{China Scholarship Council (CSC)}
  \end{itemize}
}

%%%%%%%%%%%%%%%%%%%%%%%%%%%%%%%%%%%%%%%%%
\section{Research Interests}

\cvitem{}{My research interests lie in \textbf{optimization theory and algorithms}, with a focus on their applications in \textbf{signal processing} and \textbf{wireless communications}.}

%%%%%%%%%%%%%%%%%%%%%%%%%%%%%%%%%%%%%%%%%
\section{Publications \& Preprints}

\cvitem{2025}{\textbf{Xiaozhi Liu}, Yong Xia. \textit{A Unified Algorithmic Framework for Dynamic Compressive Sensing}. Signal Processing: 232, 109926. [\href{https://github.com/xzliu-opt}{github}]}

\cvitem{2025}{\textbf{Xiaozhi Liu}, Yong Xia. \textit{Cubic NK-SVD: An Algorithm for Designing Parametric Dictionary in Frequency Estimation}. Signal Processing: 235, 110029. [\href{https://github.com/xzliu-opt}{github}]}

\cvitem{2025}{\textbf{Xiaozhi Liu}, Yong Xia. \textit{Split-Merge: A Difference-based Approach for Dominant Eigenvalue Problem}. arXiv: 2501.15131.}

\cvitem{2025}{\textbf{Xiaozhi Liu}, Yong Xia. \textit{Split-Merge Revisited: A Scalable Approach to Generalized Eigenvalue Problems}. arXiv: 2507.02389.}

\cvitem{2024}{\textbf{Xiaozhi Liu}$^*$, Jinjiang Wei$^*$, Yong Xia. \textit{Revisiting Atomic Norm Minimization: A Sequential Approach for Atom Identification and Refinement}. arXiv: 2411.08459.}

%%%%%%%%%%%%%%%%%%%%%%%%%%%%%%%%%%%%%%%%%
\section{Research Experience}

\cventry{Sep. 2022 -- Present}{Core Technical Member}{Super-Resolution Parameter Estimation in 5.5G Massive MIMO}{National Key R\&D Program of China}{}{
  \begin{itemize}
    \item Addressed issues related to the estimation of wireless channel state information (CSI) and the optimization of hybrid beamforming (HBF) algorithms in 5.5G Massive MIMO systems
    \item Developed novel algorithms for parameter estimation and signal completion
  \end{itemize}
}

\cventry{Nov. 2020 -- Jan. 2021}{Project Leader}{Application of BERT Model in Cloze Tests for NLP}{ASC International Student Supercomputer Challenge}{\textbf{Second Prize}}{
  \begin{itemize}
    \item Independently studied the BERT model under the PyTorch framework from scratch
    \item Implemented training and testing of the CLOTH dataset using Python programming
    \item Leveraged high-performance computing platform (Linux) for GPU parallel computing to enhance computational efficiency
  \end{itemize}
}

%%%%%%%%%%%%%%%%%%%%%%%%%%%%%%%%%%%%%%%%%
\section{Work Experience}

\cventry{Aug. 2025 -- Sep. 2025}{Research Assistant}{Hong Kong Baptist University}{Hong Kong, China}{}{
  \begin{itemize}
    \item Supervisors: Prof. Michael K. Ng (SIAM Fellow) and Prof. Guangning Xu
    \item Research Focus: Improvement of LoRA-based parameter-efficient fine-tuning (PEFT) strategies for large language models (LLMs)
  \end{itemize}
}

%%%%%%%%%%%%%%%%%%%%%%%%%%%%%%%%%%%%%%%%%
\section{Presentations}

\cvitem{Oct. 12-15, 2023}{\textit{A Unified Algorithmic Framework for Dynamic Compressive Sensing}. 21st Annual Meeting of CSIAM, Kunming, Yunnan.}

\cvitem{Sep. 13-15, 2024}{\textit{Cubic NK-SVD: An Algorithm for Designing Parametric Dictionary in Frequency Estimation}. 1st ORSC Conference on Data Science and Operations Research Intelligence, Beijing.}

%%%%%%%%%%%%%%%%%%%%%%%%%%%%%%%%%%%%%%%%%
\section{Honors \& Awards}

\cvitem{2025}{\textbf{National Scholarship for Doctoral Students} (Top 0.2\% nationwide)}
\cvitem{2021}{\textbf{National Scholarship for Undergraduate Students} (Top 0.2\% nationwide)}
\cvitem{2022}{\textbf{Ph.D. Freshman Scholarship} (awarded to only 3 students in the college)}
\cvitem{2022}{\textbf{Outstanding Graduate}}
\cvitem{2020}{\textbf{First Prize}, National Undergraduate Mathematics Competition, Shaanxi Province}
\cvitem{2020}{\textbf{First Prize}, China Undergraduate Mathematical Contest in Modelling (CUMCM), Shaanxi Province}

%%%%%%%%%%%%%%%%%%%%%%%%%%%%%%%%%%%%%%%%%
\section{Technical Skills}

\cvitem{Programming}{Python, Matlab, C, Julia, \LaTeX}
\cvitem{Machine Learning}{PyTorch, TensorFlow}
\cvitem{HPC}{Linux, GPU Parallel Computing}
\cvitem{Languages}{
  \textbf{English:} Fluent (CET-4: 593, CET-6: 523, PETS Level 5) \newline
  \textbf{Mandarin:} Native Speaker
}

%%%%%%%%%%%%%%%%%%%%%%%%%%%%%%%%%%%%%%%%%

\end{document}
